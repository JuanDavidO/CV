%%%%%%%%%%%%%%%%%%%%%%%%%%%%%%%%%%%%%%%%%
% Plasmati Graduate CV
% LaTeX Template
% Version 1.0 (24/3/13)
%
% This template has been downloaded from:
% http://www.LaTeXTemplates.com
%
% Original author:
% Alessandro Plasmati (alessandro.plasmati@gmail.com)
%
% License:
% CC BY-NC-SA 3.0 (http://creativecommons.org/licenses/by-nc-sa/3.0/)
%
%%%%%%%%%%%%%%%%%%%%%%%%%%%%%%%%%%%%%%%%%

%----------------------------------------------------------------------------------------
%	PACKAGES AND OTHER DOCUMENT CONFIGURATIONS
%----------------------------------------------------------------------------------------

\documentclass[a4paper,10pt]{article} % Default font size and paper size

\usepackage{fontspec} % For loading fonts
\defaultfontfeatures{Mapping=tex-text}
%\setmainfont[SmallCapsFont = Fontin SmallCaps]{Fontin} % Main document font

\usepackage{xunicode,xltxtra,url,parskip} % Formatting packages

\usepackage[usenames,dvipsnames]{xcolor} % Required for specifying custom colors

\usepackage[big]{layaureo} % Margin formatting of the A4 page, an alternative to layaureo can be \usepackage{fullpage}
% To reduce the height of the top margin uncomment: \addtolength{\voffset}{-1.3cm}

\usepackage{hyperref} % Required for adding links	and customizing them
\definecolor{linkcolour}{rgb}{0,0.2,0.6} % Link color
\hypersetup{colorlinks,breaklinks,urlcolor=linkcolour,linkcolor=linkcolour} % Set link colors throughout the document

\usepackage{titlesec} % Used to customize the \section command
\titleformat{\section}{\Large\scshape\raggedright}{}{0em}{}[\color{black} \titlerule] % Text formatting of sections
\titlespacing{\section}{0pt}{3pt}{3pt} % Spacing around sections

\usepackage{longtable}
\usepackage[spanish]{babel}

\begin{document}

\pagestyle{empty} % Removes page numbering

\font\fb=''[cmr10]'' % Change the font of the \LaTeX command under the skills section

%----------------------------------------------------------------------------------------
%	PERSONAL INFORMATION
%----------------------------------------------------------------------------------------

%\par{\centering{\LARGE \color{OrangeRed} \textsc{Curriculum Vitae}}\bigskip\par} % Your name

\par{\centering{\Huge \color{black}  Juan David} \bigskip\par} % Your name
\par{\centering{\huge \color{black} \textsc{ORJUELA Z\'U\~NIGA}}\bigskip\par} % Your name

\color{OrangeRed}
\section{Personal Information}
\color{black}

\begin{tabular}{rl}
\textsc{Teléfono:} & +57 300 481 5575 (Bogot\'a, Colombia)\\
\textsc{Email:} & \href{mailto:jd.orjuelam@uniandes.edu.co}{jd.orjuelam@uniandes.edu.co}\\
\textsc{GitHub:} & \url{github.com/juandavido}
\end{tabular}

%----------------------------------------------------------------------------------------
%	EDUCATION
%----------------------------------------------------------------------------------------

\color{OrangeRed}
\section{Educación}
\color{black}

\begin{tabular}{rl}	

\textsc{Presente} & \textsc{Estudiante de Maestría en Ciencias} - Física. Actualmente en \\ 
\textsc{January 2015} & segundo año en la \textbf{Universidad de los Andes} en Bogot\'a, Colombia. \\
& Cursos relevantes: Bioinformática, Métodos computacionales avanzados. \\
\\
\textsc{Diciembre 2014} & \textsc{Físico} de la \textbf{Universidad Nacional de Colombia} en Bogot\'a, Colombia. \\ 
\textsc{Agosto 2007} &  Promedio=3.9/5. Cursos relevantes: Métodos de simulación física,  
\\& Herramientas computacionales, Programación y métodos numéricos,
\\& Programación orientada a objetos. \\
\\
\textsc{Junio 2007}& \textsc{}\textsc{Bachiller académico} del \normalsize\textbf{Colegio Champagnat} en Popay\'an, Colombia.\\
%&\\

\end{tabular}

%----------------------------------------------------------------------------------------
%	RESEARCH EXPERIENCE
%----------------------------------------------------------------------------------------

\color{OrangeRed}
\section{Experiencia en Investigación}
\color{black}

\begin{tabular}{rl}

\textsc{Agosto 2015} & Investigador en el proyecto: ``Mechanical properties of lipid bilayers containing \\ 
\textsc{Junio 2015}   & lysolipids'' financiado por el Comité de investigación y posgrados Universidad \\ 
& de los Andes. Investigador principal: Antonio Manu Forero Shelton.  \\
& Proponer, hacer pruebas y ejecutar modelos biofísicos para determinar el \\ & conjunto de parámetros que mejor reproduce las propiedades fisicoquímicas \\& del sistema estudiado. \\
\\ 
\textsc{Junio 2015} & Investigador asistente para el proyecto: ``Estudio de Factibilidad de uso de \\ 
\textsc{Enero 2015} & detectores MEDIPIX para imágenes mamográficas'', etapa: ''Estudio de Tejidos  
\\& Blandos Animales con Microcalcificaciones Usando el Detector MEDIPIX''. \\
& Diseñar y ejecutar pruebas que corroboraron la viabilidad del uso de detectores  \\& MEDIPIX para detección temprana y segura de cáncer de seno usando \\& tejido animal. \\
\\
\textsc{Marzo 2014} & Proyecto de grado: ``Energetic analysis of the  extraction process of a phospholipid\\
\textsc{Agosto 2013} & from a lipid bilayer using Molecular Dynamics simulations'' (Director: Jos\'e Daniel \\ & Casta\~no). \\
& Extracción, procesamiento y visualización de datos para medir cambios en las \\ & propiedades de membranas a partir de las simulaciones.

\end{tabular}

%\newpage

%----------------------------------------------------------------------------------------
%	TEACHING EXPERIENCE
%----------------------------------------------------------------------------------------

\color{OrangeRed}
\section{Experiencia Docente}
\color{black}

\begin{longtable}{rl}

\textsc{Presente} & Profesor de Herramientas Computacionales en la Universidad de los Andes.\\ 
\textsc{Agosto 2015} & Enseñar y ayudar a estudiantes de Ciencia e Ingeniería a desarrollar habilidades \\& de programación en un lenguaje de alto nivel (Python), con métodos y  \\& herramientas para análisis numérico y de datos básico.  \\
\\

\textsc{Junio 2015} & Profesor de Laboratorio de Física II en la Universidad de los Andes, Bogot\'a.  \\ 
\textsc{Enero 2015} & Entrenar estudiantes de Ciencias e Ingeniería en los métodos y \\& habilidades de la física experimental.\\
\\

\textsc{Junio 2015} & Profesor de Laboratorio de Física Básica II en la Universidad de los Andes. \\ 
\textsc{Enero 2015} & Entrenar estudiantes de Ciencias de la Vida en los métodos y habilidades \\& de la física experimental.\\
\\

\textsc{Junio 2015} & Tutor de Matemáticas (bilingüe - en línea) para Latinhire Inc., Bogot\'a\\ 
\textsc{Enero 2015} & Enseñar a estudiantes universitarios y de secundaria temas variados de \\& matemáticas hasta ecuaciones diferenciales incluyendo matemáticas \\& para estudiantes de negocios y de ciencias sociales.

\end{longtable}

%----------------------------------------------------------------------------------------
%	SCHOOLS AND EVENTS
%----------------------------------------------------------------------------------------

\color{OrangeRed}
\section{Schools and Events}
\color{black}

\begin{tabular}{rl}

\textsc{Agosto 2015} & 2015 Martini Coarse-Graining Workshop, Groningen, The Netherlands \\
&\\

\textsc{Septiembre 2014} &  2nd Workshop on Statistical Physics, Bogotá, Colombia, \\
&\\

\textsc{Febrero 2014} &  58th Annual Meeting of the Biophysical Society. San Francisco, CA. \\
%&\\

\end{tabular}

%----------------------------------------------------------------------------------------
%	SERVICE
%----------------------------------------------------------------------------------------

%----------------------------------------------------------------------------------------
%	SCHOLARSHIPS AND DISTINCTIONS
%----------------------------------------------------------------------------------------

\color{OrangeRed}
\section{Distinciones}
\color{black}

\begin{tabular}{rl}

\textsc{Noviembre} 2007 & \textsc{Primer lugar} regional (Cauca) en las pruebas Saber 11 (ICFES) - 2007. \\ 
&\\

\textsc{Junio} 2007 & \textsc{Primer lugar} a nivel nacional en las pruebas Saber 11 (ICFES) - 2007-I. \\ 
&\\

\textsc{Junio} 2005 & \textsc{Primer lugar} en las pruebas regionales. XXIV Olimpiadas Colombianas  \\& de Matemáticas Nivel intermedio (competencia hasta la ronda final).  \\& Universidad Antonio Nariño.\\
%&\\

\end{tabular}

%----------------------------------------------------------------------------------------
%	LANGUAGES
%----------------------------------------------------------------------------------------

\color{OrangeRed}
\section{Lenguajes}
\color{black}

\begin{tabular}{rl}
\textsc{Español:} & Competencia nativa \\

\textsc{Inglés:} & Competencia profesional plena \\

\textsc{Francés:} & Competencia limitada de trabajo \\

\textsc{Alemán:} & Competencia básica \\
\end{tabular}

%----------------------------------------------------------------------------------------
%	COMPUTER SKILLS 
%----------------------------------------------------------------------------------------

\color{OrangeRed}
\section{Computer Skills}
\color{black}
 
\begin{tabular}{rl}
Lenguajes de programación: & C++ (2.5 años), C (1.5 años), Python (1.5 años). \\ %, Matlab (6 months). \\

Elementales: & GitHub, {\fb \LaTeX}, \textsc{UNIX}.\\

Sistemas operativos: & Ubuntu, Windows.\\

%Other software: & GROMACS, VMD, PyMOL, Origin, Excel. \\

\end{tabular}

%----------------------------------------------------------------------------------------
%	INTERESTS AND ACTIVITIES
%----------------------------------------------------------------------------------------

\color{OrangeRed} 
\section{Habilidades, intereses y aficiones}
\color{black}

\begin{tabular}{rl}	

\textsc{Habilidades:} & Analítico, intuición con los datos. Recursivo, diplomático, adaptable, con \\& muchas ganas de aprender.\\
\textsc{Intereses:} & Data Science, analytics, programación y física (biofísica y química computacional,\\&  mecánica de fluidos computacional con métodos de Lattice Boltzmann). \\
\textsc{Aficiones:} & Coro (miembro del coro universitario por 1.5 años), teatro (trabajé con una \\& compañía independiente por 4 años). \\
\end{tabular}
%----------------------------------------------------------------------------------------
\end{document}
